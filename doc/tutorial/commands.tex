
%
% Packages
% ========
%


\usepackage[latin1]{inputenc}
\usepackage[a4paper,pdftex,top=2cm,left=3.5cm,right=3.5cm,bottom=2cm]{geometry}

%\usepackage[brazil]{babel}
\usepackage{graphicx}                 % figuras
%\usepackage{floatflt}                % figuras como ``frames'' a direita
%\usepackage{wrapfig}                 % idem a floatflt, mas parece funcionar melhor
\usepackage{subfigure}

\usepackage{latexsym}                 % mais s{\'\i}mbolos matem{\'a}ticos
\usepackage{amssymb}                  % mais s{\'\i}mbolos matem{\'a}ticos
%\usepackage{oz2e}                     % Z notation
\usepackage{theorem}

\usepackage{enumerate}                % permite apresentar um exemplo de como ser{\'a} a enumera{\c c}{\~a}o
\usepackage{tabularx}
\usepackage{booktabs}                 % melhora o desenho das tabelas com \toprule \midrule e \bottomrule
%\usepackage{showkeys}                % mostra os ref..... op{\c c}{\~o}es: [notref,notcite,final]

%\usepackage{makeidx}                  % {\'\i}ndice
%\makeindex
\usepackage{verbatim}


\RequirePackage[bf,center]{caption}
\RequirePackage{color}
\RequirePackage{url}
\RequirePackage{hyperref}        % PDF com links (pode-se usar como opção: dvipdfm, dvips, ...)
\hypersetup{colorlinks=true,urlcolor=blue,linkcolor=blue,citecolor=blue} % colorido
\RequirePackage{setspace}


\usepackage{algorithmic}
\usepackage[vlined, linesnumbered]{algorithm2e}

\newenvironment{remark}{\medskip\noindent \textsc{Remarks.}} {\medskip}


%\usepackage{listings}                 % for java listings
%\lstset{basicstyle=\footnotesize\sffamily,labelstep=1,labelstyle=\tiny,indent=1.2cm,extendedchars=true,frame=}
%\lstset{basicstyle=\footnotesize\sffamily,extendedchars=true,frame=}

\usepackage{xspace}

\usepackage{prettyref}                % facilita a constru{\c c}{\~a}o de refer{\^e}ncias, deve-se usar \prettyref{...}
\newrefformat{fig}{Figure~\ref{#1}}
\newrefformat{tab}{Table~\ref{#1}}
\newrefformat{sec}{Section~\ref{#1}}
\newrefformat{eq} {equation~(\ref{#1})}
\newrefformat{chp}{Chapter~\ref{#1}}
\newrefformat{ex} {example~\ref{#1}}
\newrefformat{ap} {appendix~\ref{#1}}
\newrefformat{alg}{algorithm~\ref{#1}}


%\usepackage{abnt-alf} % para as ref. no formato ABNT

\usepackage[sort]{natbib} % melhora as referencias (citep, citet, citeauthor, ...)
%\bibpunct{(}{)}{e}{a}{,}{,}

\usepackage{acronym}

\sloppy

% Commands
% ========
%
\newcommand{\email}[2]{\href{mailto:#2}{#1}}
\renewcommand{\and}[0]{\\\vspace{6pt}}

\newcommand{\srctutorial}[0]{../../examples/tutorial}
\newcommand{\srcwritepaper}[0]{../../examples/writePaper}
\newcommand{\wpjason}[0]{jmoise/example/writePaper}

\newcommand{\toindex}[1]{\index{#1}\marginpar{\scriptsize{\textsf{i:#1}}}}
\newcommand{\indexdef}[1]{\index{#1|textsl}\marginpar{\scriptsize{\textsf{id:#1}}}}

\newcommand{\defEventoOrg}[5]{
  \vspace{.5cm}
  \noindent
  \begin{tabular}{l l}
    \toprule
    Evento         & #1 \\
    \midrule
    Argumentos     & #2 \\
    \midrule
    Pr{\'e}-condi{\c c}{\~o}es  & #3 \\
    \midrule
    Efeitos        & #4 \\
    \bottomrule
  \end{tabular}
  \vspace{.5cm}

  \noindent #5
}


\newcommand{\tdef}[0]{\; =_{\textrm{\tiny{def}}} \;} %stackrel{\triangle}{=}
\newcommand{\tq}[0]{\;\;\textrm{tal que}\;\;}
\newcommand{\subrole}[0]{\sqsubset}

\newcommand{\ie}[0]{isto {\'e}}
\newcommand{\eg}[0]{por exemplo}

\theorembodyfont{\rmfamily}
\newtheorem{tdefn}{Defini{\c c}{\~a}o}[chapter]
\newtheorem{texplo}{Exemplo}[chapter]

\newenvironment{defn}[1]
   {\begin{tdefn}\label{def:#1}}
   {\hfill$\Box$\end{tdefn}}
\newenvironment{definicao}[2]
   {\begin{tdefn}\label{def:#2}\textbf{(\textsl{#1})}}
   {\hfill$\Box$\end{tdefn}}
\newenvironment{explo}
   {\begin{texplo}}
   {\hfill$\Box$\end{texplo}}


\newcommand{\pendencia}[1]{~[\textsc{Pend{\^e}ncia}\footnote{\textbf{#1}}]~}
%\newcommand{\pendencia}[1]{}

\newcommand{\processo}[1]{\textsc{#1}}
\newcommand{\RETURN}[0]{\textbf{return}\xspace}
\newcommand{\evento}[1]{\textsl{#1}}
\newcommand{\papel}[1]{\textsl{#1}}
\newcommand{\relacao}[1]{\textsf{#1}}
%\newcommand{\pode}[2]{#1 \relacao{pode} #2}
%\newcommand{\deve}[2]{#1 \relacao{deve} #2}
\newcommand{\pode}[0]{permission}
\newcommand{\deve}[0]{obligation}
\newcommand{\compat}[0]{\bowtie}

%\newcommand{\compat}[2]{#1 \relacao{compat} #2}

\newcommand{\group}[1]{\texttt{#1}}
\newcommand{\role}[1]{\textsf{#1}}
\newcommand{\agent}[1]{$#1$}
\newcommand{\goal}[1]{\textsl{#1}}

\newcommand{\igroup}[0]{grupo\xspace}
\newcommand{\igroups}[0]{grupos\xspace}
\newcommand{\tgroup}[0]{especifica{\c c}{\~a}o de grupo\xspace}
\newcommand{\tgroups}[0]{especifica{\c c}{\~a}o de grupos\xspace}


\newcommand{\Taems}[0]{{\sc T{\ae}ms}\xspace}
\newcommand{\taems}[0]{{\sc t{\ae}ms}\xspace}
\newcommand{\aalaadin}[0]{{\sc Aalaadin}\xspace}
%\newcommand{\Moise}[0]{{\sc Moise}\xspace}
%\newcommand{\moise}[0]{{\sc moise}\xspace}

% o \textsc n{\~a}o funcina dentro de t{\'\i}tulos
\newcommand{\moise}[0]{$\mathcal{M}${\sc{oise}}\xspace}
\newcommand{\moisem}[0]{$\mathcal{M}${\sc{oise}}$^{+}$\xspace}
\newcommand{\smoisem}{\mbox{$\mathcal{S}$-\moisem}\xspace}
\newcommand{\jmoisem}{\mbox{$\mathcal{J}$-\moisem}\xspace}
\newcommand{\saci}[0]{\textsc{Saci}\xspace}
\newcommand{\Jason}{\textit{\textbf{Jason}}\xspace}
\hyphenation{Jason}



% \renewcommand{\maketitle}{
%   \newpage
%   \thispagestyle{empty}
%   \null
%   \begin{center}


%     \begin{spacing}{2}
%       \vspace{\stretch{1}}
%       \textbf{\huge \@title}
%     \end{spacing}
%     \vspace{1cm}

%     \@author
%     \vspace{24pt}

%     %\@institution

%     \vspace{\stretch{2}}
%     %\@place \\
%     \@date

%   \end{center}

%   \hypersetup{pdftitle=\@title,pdfauthor=\@author}
% }
